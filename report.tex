\documentclass[a4paper, 10pt]{oblivoir}

\usepackage{fapapersize}
\usefapapersize{210mm, 297mm, 30mm, 30mm, 30mm, 30mm}

\usepackage[hangul]{kotex}
\usepackage{indentfirst}
\usepackage{makecell}

\usepackage{tikz}
\usetikzlibrary{graphs, graphdrawing}
\usegdlibrary{layered}

\title{컴파일러 프로젝트 1 결과보고서}
\author{박건}
\date{2019년 3월 24일}

\begin{document}

\maketitle

\vspace*{\fill}

\begin{center}
\begin{tabular}{ l l }
과목명 & [CSE4120] 기초 컴파일러 구성 \\
담당교수 & 서강대학교 컴퓨터공학과 정성원 \\
개발자 & 박건 \\
개발기간 & 2019.3.24 - 2019.3.24 \\
\end{tabular}
\end{center}

\vspace*{\fill}

\pagebreak

\begin{tabular}{ l l }
프로젝트 제목: & \makecell{Design and Development of Compiler for C- Language: \\
Phase 1: Design and Implementation of Lexical Analyzer} \\
제출일: & 2019. 3. 24.\\
개발자: & 박건 \\
\end{tabular}

\section{개발 목표}
C- Language의 Lexical Analyzer를 구현한다. C- Language는 C Language를 간소화한 언어이다.

\section{개발 범위 및 내용}
\subsection{개발 범위}
Lexical Analyzer는 Input Character들을 Token으로 쪼개는 역할을 하며, 동시에 Comment를 제거하는 기능을 한다.

\subsection{개발 내용}
\texttt{flex}를 이용하여 C- Language의 Lexical Analyzer를 구현한다.

\section{추진 일정 및 개발 방법}
\subsection{추진 일정}
3월 24일: 개발, 결과 보고서 작성
3월 27일: 과제 제출

\subsection{개발 방법}
Linux Desktop 환경에서 Qt Creator IDE를 이용하여 개발한다.

\section{연구 결과}
\tikz \graph [layered layout]
{
"\texttt{tiny.l}" -> [densely dashed] {"\texttt{lex.yy.c}", "\texttt{lex.yy.h}"},
"\texttt{globals.h}" -> {"\texttt{main.c}", "\texttt{utils.c}", "\texttt{tiny.l}"},
"\texttt{lex.yy.h}" -> {"\texttt{main.c}"}
};

\begin{itemize}
\item \texttt{tiny.l} -- \texttt{flex} Input file. 토큰들의 Regular expression 정의와 Comment parsing 알고리즘이 들어간다.
\item \texttt{globals.h} -- \texttt{enum Token}의 정의와 Enumeration 값을 문자열로 바꿔 주는 \texttt{enum\_to\_string()} 함수 선언, \texttt{struct Scanner}의 정의가 들어간다.
\item \texttt{utils.c} -- \texttt{enum\_to\_string()} 함수 정의가 들어간다.
\item \texttt{main.c} -- \texttt{main()} 함수가 들어간다.
\item \texttt{lex.yy.c} -- \texttt{tiny.l}에서 flex가 생성해준 파일로, 문자열을 파싱해주는 \texttt{yylex()} 함수가 정의되어 있다.
\item \texttt{lex.yy.h} -- \texttt{tiny.l}에서 flex가 생성해준 헤더 파일로, \texttt{lex.yy.c}에 정의되어 있는 여러 함수들이 선언되어 있다.
\end{itemize}



\section{기타}



\end{document}